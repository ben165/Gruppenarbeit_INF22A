\subsection{Beurteilung und Reaktion auf die Online-Präsenz der Konkurrenz}

\subsubsection*{Erklärung}

Die Online-Präsenz eines Unternehmens ist heute mit die wichtigste Werbeplattform. Weltweit wird das Internet inzwischen von 95 \% der Menschen genutzt \cite{internetnutzerBU}. Eine Konkurrenz-Analyse in diesem Bereich ist besonders wichtig, auch deshalb, weil Dienstleistungen und Produkte im Smartphone-Zeitalter immer öfters online gesucht werden.

Es beschränkt sich aber nicht nur auf die Werbung, im Falle von Social Media ist es wichtig, regelmäßig neuen Content zu liefern, um Kunden zu gewinnen und auf Kritik zu reagieren, damit sich die Käufer weiterhin mit dem Unternehmen identifizieren können. So musste der Hersteller der bekannten \textit{Böhme Fruchtkaramellen} nach einer Designänderung wieder auf das ursprüngliche Design zurückkehren, nachdem über Social-Media-Kanäle zu viel Beschwerden eingingen \cite{fruchtkaramellenBU}.

\subsubsection*{Beurteilung und Reaktion}

Die Möglichkeiten einer Online-Präsenz sind heute breit gefächert. Beginnend mit der Internetseite und der eigenen Domain, über Social-Media-Kanäle wie Instagram, Facebook und TikTok gibt es auch noch die Online-Jobbörsen, um neue Arbeitnehmer auf sich aufmerksam zu machen. Marktplätze wie Amazon, Ebay und Steam ermöglichen es, Produkte zu verkaufen, ohne eigene Infrastruktur in Form eines Online-Shops.

Ergänzend dazu helfen Suchmaschinen und Kartendienste die Angebote überhaupt erst zu finden. In manchen Fällen kann auch eine eigene App Sinn ergeben.

Das Dokument beschäftigt sich mit der Frage, wie die einzelnen Internetauftritte der Konkurrenz bewertet werden können. Es bietet darüber hinaus Hinweise, wie darauf reagiert werden kann. Am Ende weiß der Leser, wie er seinen eigenen Internetauftritt gestaltet und welche Plattformen wichtig sind.

